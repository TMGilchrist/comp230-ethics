% Please do not change the document class
\documentclass{scrartcl}

% Please do not change these packages
\usepackage[hidelinks]{hyperref}
\usepackage[none]{hyphenat}
\usepackage{setspace}
\doublespace

% You may add additional packages here
\usepackage{amsmath}

% Please include a clear, concise, and descriptive title
\title{To What Extent is it Ethical to Give Players the Ability to Kill People in a Video Game?}

% Please do not change the subtitle
\subtitle{COMP230 - Ethics and Professionalism}

% Please put your student number in the author field
\author{1604281}

\begin{document}
	
	\maketitle
	
	\abstract{}
	
	\section{Introduction}
		A large number of video games depict violence, often against other humans, and many of these games actively encourage, reward and support the player in committing these actions themselves. \cite{esaReport} The psychological impact of playing these violent games is an ongoing and controversial discussion that relates to the overall issue of violence in all forms of media, and the effects it can have. Unlike other forms of media, video games allow the audience to partake in - and instigate - violent actions. With a growing audience of over two billion gamers worldwide as of 2017, there is understandable concern over how exposure to such games could affect such a huge part of the world's population. \cite{numberOfGamers}
		
	
	\section{Potential Issues}
		One common argument against violent video games is that they can desensitise players to their violent content, making them less averse to acts of violence and therefore more likely to go on to commit violent crimes. \cite{devalueViolence} \cite{ExposureLink}
		
		Similarly, it has also been suggested that playing violent games can directly increase the players' aggression levels, contributing to short-term psychological effects such as a pro-violent attitudes. \cite{anderson2007violent}
	
	
	\section{Discussion}
		However, there has been evidence to show that violence and competitiveness in games are two separate components with differing effects on the player. One study found that competitive elements in video games made players far more aggressive than games that were simply violent, with a less competitive nature. \cite{Competativness} This suggests that violence is not particularly harmful in games and is therefore more morally acceptable for developers to implement, although another study found that when playing competitive games, the more violent competative games also led to higher levels of player agression. \cite{ViolenceSports} This raises concerns over highly competitive multilayer games which often combine violence against humans and competitiveness, leading to a scenario in which the player - feeling aggressive - associates aggression and frustration with killing humans. 
		
		It is interesting to note that violent and competitive games such as rugby and hockey have been a part of many societies for over a hundred years, and have resulted in injuries and deaths during riots among supporters. Despite this, they are accepted as an integral part of many cultures, largely due to the balance of risk against entertainment and financial value. \cite{wrongToPlayGames} There is clearly a perceived moral acceptance of violence when it facilitates entertainment which is highly supported by the large number of people who play violent games. The acceptance of violence and competition in a sports environment also raises the subject of context and its effect on the audience's experience of violence. 
		
		The context of violence in games can be crucial to the effect is has on the player. A power-fantasy game such as Halo uses violence to create feelings of strength and power, while narrative-driven horror game The Last of Us elicits emotional responses from many players. CITE HERE This implies that the values expressed by a game have the potential to influence the perception of its content. \cite{ValueViolence} If this is the case, then the effect that violent games have on players can be attributed to the way in which the developers present the violence, and raises ethical concerns over the moral duty of developers to present meaningful values to counteract violent content. However it could be argued that pushing moral values in all violent games would only alienate a large proposition of a game's target market by limiting the entertainment value of the game.
		
		Even without a deliberately created set of values or morals in a game, the player may not necessarily be influenced by violent content. While many games allow the player to commit violent acts, there is an important distinction between the player morally endorsing those actions. A player may kill someone in a game while understanding that it is wrong, and merely viewing the actions from the character's point of view. \cite{FreeWill}
	
	\section{Conclusion} (TOO LONG)
		Overall, it is hard to state the effect of violent games conclusively, as there is evidence on both sides of the debate. However, as a form of entertainment, games must balance their risks with the value that can be gained from them, as well as the desires of their audience. While it could be argued that developers are responsible for the way they portray violence in their games - and therefore the way in which it effects the player - to burden the developer with the responsibility for any potential effects of violence would not greatly improve the issue, as it would merely drive players away from those developers who acted on this responsibility. Without stricter legislation, there will always be developers who cater to the market for violent games. Furthermore, it is ultimately the audience whose tastes dictate the content of games through their purchases. Despite this, developers still have a moral duty to avoid portrays that trivialise more extremely serious acts of violence and depravity. In conclusion, it is largely ethical to give players the ability to kill people in games - as it is their own choice to play it - however some care should be taken to present violence in more acceptable fashions.
	

	\bibliographystyle{IEEEtran}
	\bibliography{references}
	
\end{document}