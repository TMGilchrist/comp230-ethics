\documentclass{beamer}
\usetheme{Madrid}
\usepackage[english]{babel}
\setbeamertemplate{navigation symbols}{}
\setbeamertemplate{frametitle}[default][center]
\setbeamertemplate{bibliography item}{\insertbiblabel}
\usefonttheme[onlymath]{serif}

\usepackage{graphicx}
\graphicspath{ {./images/} }
\usepackage{amsmath}
\usepackage{wrapfig}

\title{Research Presentation}
\subtitle{COMP230}
\author{1604281}

\begin{document}

\begin{frame}
	\maketitle
\end{frame}

\begin{frame}{Killing People in Video Games}
    \begin{itemize}
        \item Many video games depict violence, often against other people. 
        
        \item Over 25 percent of the best selling games in 2017 were classed as Shooters, which is a typically violent genre. \cite{numberOfGamers}
        
        \item With over 2.2 billion gamers worldwide in 2017, games have the ability to reach a huge audience. \cite{esaReport}
        
    \end{itemize}
\end{frame}

\begin{frame}{Issues}


    \begin{itemize}    
    	\item Ongoing debate about the psychological impact that violence in video games can have on players, which is itself related to the greater issue of violence in media in general. \cite{ExposureLink}
    	    
        \item Video games can desensitise their audience to violence, thereby making them less disinclined to towards it. \cite{devalueViolence}
        
        \item Violent video games can increase levels of aggression in players. \cite{anderson2007violent}        
        
    \end{itemize}

\end{frame}

\begin{frame}{Relevance in the Games Industry}
    Several aspects of these issues are of particular relevance to the games industry: 
    
    \begin{itemize}
        \item The legality of games regarding their violent content - some games are banned or heavily censored. (e.g. Manhunt 2, Postal 2)
        
        \item Morality of working for or endorsing a company that produces violent content.
        
        \item Responsibility as an independent developer.
        
    \end{itemize}
\end{frame}

\begin{frame}{Research Question}

	\textbf{Is it Ethical to Give Players the Ability to Kill People in a Video Game?} \newline
	    
    \begin{itemize}
    	\item To what extent is it ethical to create games that allow or encourage acts of violence against human characters?
    	
        \item Where does responsibility for the violent content of a video game lie? With the player for choosing to engage with it, or the developers for providing the means for the player to experience it?        
    \end{itemize}
\end{frame} 

\begin{frame}{Discussion}

	\begin{itemize}
		\item Violent games have been prevalent in our culture in forms such as rugby and hockey for over a hundred years. The risk of games must be balanced against their value as forms of entertainment, training and revenue. \cite{wrongToPlayGames}  
		
		\item Violence and competitiveness are two separate components of games that are often closely linked. There is evidence that the competitive nature of games can have a far greater effect on player aggression than violent content. \cite{Competativness}	
		
		\item Despite this, playing violent video games has been linked to short term effects such as increases in pro-violent attitudes \cite{anderson2007violent}		
	\end{itemize}
\end{frame}

\begin{frame}{Discussion Continued}
	\begin{itemize}
		\item The values expressed by the game can change the way in which its violent content is perceived. \cite{ValueViolence}
		
		\item Should developers therefore have a moral obligation to portray violence in a weighty and non-trivial fashion?	
		
		\item Argument that players can perform violent or immoral acts in a video game without endorsing those actions themselves. \cite{FreeWill}
	\end{itemize}
\end{frame}


\begin{frame}{Difficulties}
	\begin{itemize}
		\item Morality and ethics are both subjective and hard to quantify. 
		\item Difficult to measure effects of video games on players' aggression or reaction to violence. 
	\end{itemize}
\end{frame}

\begin{frame}[allowframebreaks]{References}
	\bibliographystyle{ieeetran}
	\bibliography{references}
\end{frame}

\end{document}
